% PROJECT: <ETD> Electronic Thesis and Dissertation Initiative
%   TITLE: LaTeX report template for ETDs in LaTeX
%  AUTHOR: Neill Kipp, nkipp@vt.edu
%     URL: http://etd.vt.edu/latex/
% SAVE AS: etd.tex
% REVISED: September 6, 1997 [GMc 8/30/10]
% 

% Instructions: Remove the data from this document and replace it with your own,
% keeping the style and formatting information intact.  More instructions
% appear on the Web site listed above.

%\documentclass[12pt,dvips]{report}
\documentclass[12pt]{report}

\setlength{\textwidth}{6.5in}
\setlength{\textheight}{8.5in}
\setlength{\evensidemargin}{0in}
\setlength{\oddsidemargin}{0in}
\setlength{\topmargin}{0in}

\setlength{\parindent}{0pt}
\setlength{\parskip}{0.1in}

% Uncomment for double-spaced document.
% \renewcommand{\baselinestretch}{2}

% \usepackage{epsf}

% BHH Packages
\usepackage[]{mcode}
\usepackage[retainorgcmds]{IEEEtrantools}
\usepackage{amsmath}
\usepackage{amssymb}
\usepackage{graphicx,dblfloatfix}
\usepackage{subfig}


\begin{document}

\newcommand{\fourier}{\mathcal{F}}

\thispagestyle{empty}
\pagenumbering{roman}
\begin{center}

% TITLE
{\large
Simulation of Various Channelizer Structures 
Directed by Cyclostationary Detector}

\vfill

Brian H. Hulette

\vfill

Project Report submitted to the Faculty of the \\
Virginia Polytechnic Institute and State University \\
in partial fulfillment of the requirements for the degree of

\vfill

Master of Engineering \\
in \\
Electrical Engineering

\vfill

Amir I. Zaghloul, Co-Chair \\
Jeffrey H. Reed, Co-Chair \\
T. Charles Clancy

\vfill

February 16, 2015 \\
Falls Church, Virginia

\vfill

Keywords: SDR, Cyclostationarity, Detection, Channelizer
\\
Copyright 2015, Brian H. Hulette

\end{center}

\pagebreak

\thispagestyle{empty}
\begin{center}

{\large Simulation of Various Channelizer Structures 
Directed by Cyclostationary Detector}

\vfill

Brian H. Hulette

\vfill

(ABSTRACT)

\vfill

\end{center}

One common problem in Software-Defined Radio (SDR) systems is the problem of
detecting and then isolating narrow signals of interest in wideband sampled
data. This involves using some means to detect the frequency of one or more
signals of interest and then tuning, filtering, and decimating them. This
problem is particularly challenging due to the high sample rate of this
wideband data, so efficient algorithms are very desirable. Solutions to this
problem have applications in many areas, including both Cognitive Radio and
Electronic Warfare.

A MATLAB simulation of an SDR framework for detecting and sub-band tuning
channelized signals is presented, with a focus on finding an efficient means to
combine the two algorithms. Detection is performed using the
spectral-correlation density function which exploits the cyclostationary
property of digital signals \cite{Gardner1}.  These detections are then used to
direct a channelizer which will tune, filter, and decimate all of the detected
signals. Two different channelizer structures are examined – a polyphase
analysis/synthesis channelizer \cite{Harris1} and an overlap-save filter bank
\cite{Borgerding1}.

A summary of relevant background information is provided.  This includes
a discussion of cyclostationarity and the spectral-correlation density function
as well as background on both channelizer structures. For the overlap-save
filter bank it will be shown that in many cases computation can be saved by
using the same FFT computation for both the cyclostationary detector and the
channelizer.

In this simulation, simple QPSK signals are used to model the signals of
interest, but the framework is applicable to any modulation or signal type that
has cyclostationary features. While other means of signal detection may be more
reliable for specific signals, a cyclostationary detector's primary strength is
its ability to function for many different types of digital signals.


\vfill

% GRANT INFORMATION

%That this work received support from the Southeastern Universities
%Research Association (SURA) ``Monticello Library Project'' is purely
%coincidental.

\pagebreak

% Dedication and Acknowledgments are both optional
% \chapter*{Dedication}
\chapter*{Acknowledgments}
I would like to acknowledge the work of both Craig Carlson and Ruth Stoehr.
Much of the source for this project for both generating and processing a simple
QPSK signal came from the ECE5654 project which we completed in the Spring of
2013.

I would like to thank my current employer, n~ask, inc., and all of my
co-workers there for their support.

I would also like to thank the members of my committee, particularly Dr.
Zaghloul, for their seemingly endless patience and support.

Finally I would like to thank my friends and family, particularly my girlfriend
Meredith, for letting me use ``My Masters" as an excuse for nearly anything.

\tableofcontents
\pagebreak

\listoffigures
\pagebreak

\listoftables
\pagebreak

\pagenumbering{arabic}
\pagestyle{myheadings}

\chapter{Introduction}
\label{sec:intro}
\markright{Brian H. Hulette \hfill Chapter \ref{sec:intro}. Introduction \hfill}

A common problem when designing a Software-Defined Radio (SDR) system is the
problem of detecting and tuning signals of interest. A typical SDR will sample
at a relatively high rate to acquire a wideband signal. Then it will attempt to
detect the channels within the acquired frequency range that contain signals of
interest, and then tune, filter, decimate, and demodulate those signals. 

%TODO: links to SDR frameworks (webSDR) TODO: image of SDR plotting
Many popular SDR applications perform this detection step with
a "man-in-the-loop".  A falling raster of the wideband data is presented and
the man-in-the-loop will "click-tune" to select  the frequencies that (s)he
would like to process. One common version of this is to acquire a wideband
slice of the HAM radio spectrum and then click tune to different frequencies to
demodulate new FM Push-to-Talk signals and listen to them.

In this paper I will investigate and simulate a structure that automates this
process - but rather than working for a single analog signal, this framework is
intended for an arbitrary number of digital signals. Detection is performed by
a cyclostationary detector rather than by a man-in-the-loop, in order to
exploit the cyclostationary properties exhibited by most digital signals with
a fixed symbol rate.  Tuning, filtering, and decimating is performed by
a channelizer, so that many signals can be processed at once.  I will examine
two different channelizer structures. They will be compared on two important
criteria:
1) Their ability to accurately reproduce every detected signal, and 2) Their
computational efficiency when combined with a cyclostationary detector.
% TODO: list papers that discuss SCD for various protocols

In Chapter~\ref{sec:cyclo} I will provide background information on
cyclostationary properties and the cyclic spectra upon which my detector
relies.  Chapter~\ref{sec:chan} discusses and compares both of the channelizer
structures that I will simulate. Chapter~\ref{sec:sim}  discusses the MATLAB
simulation I have created and the results of it. Finally,
Chapter~\ref{sec:source} provides annotated source code for key parts of the
simulation, as well as locations where the entire source code can be found.

\chapter{Cyclostationary Detection}
\label{sec:cyclo}
\markright{Brian H. Hulette \hfill Chapter \ref{sec:cyclo}. Cyclostationary Detection \hfill}

\section{Cyclostationarity}
\label{sec:cyclo_prop}
Often signals are modeled as stochastic random processes with properties such as
mean and variance, but many man-made signals, particularly digital communications,
can be modeled with another statistical property called
\emph{cyclostationarity} which implies the signal has some parameter which
varies with time. The frequency of this variation is called the cyclic
frequency, $\alpha$. \cite{Gardner1}.

% TODO: make a real bibliography
% Gardner1 = Exploitation of Spectral Redundancy in Cyclostationary Signals
% Oner1 = Cyclostationarity Based Air Interface Recognition for Software Radio Systems
% Gardner2 = Spectral Correlation of Modulated Signals: Part II - Digital Modulation
% Harris1 = Digital Receivers and Transmitters Using Polyphase Filter Banks for Wireless Communications
% Borgerding1 = Turning Overlap-Save into a Multiband Mixing, Downsampling Filter Bank

Of particular interest is second-order cyclostationarity, where a signal has a
periodic autocorrelation:

\begin{IEEEeqnarray*}{lCl}
    R_{xx}(t, t+\tau) = E\{x(t)x^*(t+\tau)\} = \sum_{\alpha} R_{xx}^{\alpha}(\tau)e^{j2 \pi \alpha t}
\end{IEEEeqnarray*}

% TODO more detail here based on Oner1

\begin{IEEEeqnarray*}{lCl}
    R_{xx}^{\alpha}(\tau) = \lim_{T \to \infty} \frac{1}{T}\int_{-T/2}^{T/2} R_{xx}(t, t+\tau)e^{-j2\pi \alpha t} dt
\end{IEEEeqnarray*}
% TODO: reference papers that perform time domain detection (listed in "Energy Efficient Spectrum Sensing using Cyclostationarity")
% "Cyclostationary Based Air Interface Recognition For Software Radio Systems" is also a time domain approach
The cyclic autocorrelation function (CAF), given by $R_{xx}^{\alpha}(\tau)$,
can be used for signal detection in the time domain. An estimate of the 2D
function versus $\alpha$ and $\tau$ can be computed, then features within this
plane can be detected. However, what I am interested in is the fourier
transform of the CAF, called the Spectral Correlation Density (SCD):

\begin{IEEEeqnarray*}{lCl}
    S_{xx}(\alpha, f) = \int_{-\infty}^{\infty} R_{xx}^{\alpha}(\tau)e^{-j2\pi f \tau} d\tau
\end{IEEEeqnarray*}

The SCD is a generalization of the conventional PSD, which it reduces to at
$\alpha=0$ \cite{Oner1}. Like the CAF, the SCD also contains unique features
based on the modulation type, symbol rate, and carrier frequency of the
transmitted signal, which we can exploit for signal detection. In
\cite{Gardner2}, Gardner defines this SCD for various basic modulation types,
including QPSK, which is used in my simulation.

%\section{Exploiting Cyclostationarity for Detection}
%\label{sec:exploit_cyc}
%For the problem defined in Chapter \ref{sec:intro} we can define the wideband
%signal $W(t)$ as the sum of $N$ cyclostationary signals, tuned to
%various frequencies, in an AWGN channel:
%\begin{IEEEeqnarray*}{lCl}
%    W(t) & = & \sum C_{i}(t) e^{j 2 \pi f_i t} + N(t)
%\end{IEEEeqnarray*}
%
%Where the $C_i(t)$ are cyclostationary random processes at various symbol rates,
%tuned to corresponding frequencies $f_i$ and $N(t)$ is the AWGN.
%
%We would like to exploit the cyclostationarity of the $C_i$ waveforms to
%identify the frequency, $f_i$, of those waveforms which are transmitted at a
%specific symbol rate.

\section{Estimating the SCD}
% Use Gardner1 to show that SCD is conjugate multiplication of frequency shifted spectra
% TODO:
% - cyclic wiener relation on Gardner1 pg. 153
% - attempts to derive in lab notebook pg. 20-21
Thus we can compute a discrete slice of the SCD at cyclic freqency $\alpha$ as
the cross spectral density of the two frequency-shifted time sequences
$x_L(t) = x(t)e^{j\pi\alpha t}$ and $x_U(t) = x(t)e^{-j\pi\alpha t}$. The cross
spectral density of these two sequences is equivalent to the conjugate
multiplication of their individual fourier transforms:

\begin{IEEEeqnarray*}{lCl}
    S_{xx}(\alpha, f) & = & X_L(f)X_U^* (f) \\
                      & = & X(f - \alpha/2)X^*(f + \alpha/2)
\end{IEEEeqnarray*}

% Discuss possibility of circular shifting a forward FFT to estimate SCD slice
This means that we can compute $S_{xx}(\alpha, f)$ using a single forward FFT
by shifting the spectrum in either direction by $\alpha/2$, and then conjugate
multiplying the two shifted spectra together. This is significant, since
a simple forward FFT of the acquired data is useful for other operations, such
as channelization, as we will see in Section \ref{sec:poly_chan}.

Of course there is a caveat. In order to perform this shift accurately using an
FFT we need to circular shift the FFT by an integer number of bins.  This
means we can only accurately perform this frequency shift when $\alpha/2$ is
a multiple of the FFT resolution.  Thus $\alpha$ must satisfy the following
relationship:
\begin{IEEEeqnarray*}{lCl}
    \alpha = \frac{2kf_s}{N_{fft}} \text{, } k \in \mathbb{Z}
\end{IEEEeqnarray*}

If an estimate of the SCD at a cyclic frequency that does not satisfy this
relation is required, then we must perform the frequency shift in the time
domain to be perfectly accurate. Alternatively, we can still use a single
forward FFT and approximate the shift by interpolating between bins, but this
is only an approximation.


\chapter{Channelizer Structures}
\label{sec:chan}
\markright{Brian H. Hulette \hfill Chapter \ref{sec:chan}. Channelizer Structures\hfill}
\section{Polyphase Analysis/Synthesis Channelizer}
\label{sec:poly_chan}

\section{Overlap-save Filter Bank}
\label{sec:filter_bank}
The ``overlap-save filter bank" structure tht I use is based entirely on
a description by Mark Borgerding from March 2006 \cite{Borgerding1}.
Borgerding's concept is based on the well known Overlap-Save fast convolution
technique.

% TODO cite Borgerdings [1]-[5] here for Overlap-Save?
OS fast convolution can be used to speed up convolution with
a filter that has a long impulse response. The concept is that rather than
convolving in the time-domain - complexity $O(N^2)$ - it is faster to first
perform an FFT of both the signal and the filter and multiply in the frequency
domain, then IFFT to go back to the time domain - complexity $O(N\log_2N)$.
There is nothing new about this idea, but Borgerding's innovation is that he
shows how to extend this concept to tune, filter and decimate any number of
channels with arbitrary frequencies and bandwidths.

\emph{Tuning:} If the frequency of the signal of interest corresponds to one of
the FFT bins then we can simply circular shift the FFT output to place that bin
at the center. Then filtering can be performed at baseband.  If all channels
satisfy this criteria then we can re-use a single forward FFT for every
channel, simply by circular shifting it to the appropriate bin in every case.

However, if this condition is not satisfied then we will need to perform the
frequency shift in the time domain by multiplying by a complex phasor.
Fortunately, there is a still a way to re-use the forward FFT in this case.
After taking an FFT of the non frequency-shifted signal we can shift the
baseband filter response up to the desired frequency, then after the filtered
signal is IFFT'd back into the time domain, we perform the frequency shift with
the complex phasor.  This has the added benefit that the frequency shift is
performed after decimation.


\begin{figure}[h!]
\centerline{
    \subfloat[ARQ]{\includegraphics[width=0.45\textwidth]{overlap_save_shift}%
    \label{fig:overlap_save_shift}}
    \hfill
    \subfloat[HARQ Type I]{\includegraphics[width=0.45\textwidth]{overlap_save_time_domain}%
    \label{fig:overlap_save_time_domain}}
}
\caption{HARQ Techniques}
\label{fig:arq_types}
\end{figure*}

\chapter{Simulation}
\label{sec:sim}
\markright{Brian H. Hulette \hfill Chapter \ref{sec:sim}. Simulation \hfill}

\chapter{Results}
\label{sec:results}
\markright{Brian H. Hulette \hfill Chapter \ref{sec:results}. Results \hfill}

%%%%%%%%%%%%%%%%%
%
% Include an EPS figure with this command:
%   \epsffile{filename.eps}
%

%%%%%%%%%%%%%%%%
%
% Do tables like this:

 \begin{table}
 \caption{The Graduate School wants captions above the tables.}
\begin{center}
 \begin{tabular}{ccc}
 x & 1 & 2 \\ \hline
 1 & 1 & 2 \\
 2 & 2 & 4 \\ \hline
 \end{tabular}
\end{center}
 \end{table}

%%%%%%%%%%%%%%%%%%%%%%%%%%%%%%%%

% If you are using BibTeX, uncomment the following:
% \thebibliography
%
% Otherwise, uncomment the following:
% \chapter*{Bibliography}

% \appendix

% In LaTeX, each appendix is a "chapter"
\chapter{Project Source}
\label{sec:source}
\markright{Brian H. Hulette \hfill Chapter \ref{sec:source}. Project Source \hfill}
% TODO: make sure this is a link - and make the project public!
The full source is hosted on GitHub (\texttt{https://github.com/TheNeuralBit/cyclo\_channelizer}), but I will include some of the critical files below.

\section{\texttt{cyclic\_spectrum.m}}
\lstinputlisting{../cyclic_spectrum.m}
\section{\texttt{synthesis\_channelizer.m}}
\lstinputlisting{../synthesis_channelizer.m}
\section{\texttt{analysis\_channelizer.m}}
\lstinputlisting{../analysis_channelizer.m}
\section{\texttt{overlap\_save\_channelizer.m}}
\lstinputlisting{../overlap_save_channelizer.m}

\end{document}
