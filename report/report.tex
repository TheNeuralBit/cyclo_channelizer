%
% PROJECT: <ETD> Electronic Thesis and Dissertation Initiative
%   TITLE: LaTeX report template for ETDs in LaTeX
%  AUTHOR: Neill Kipp, nkipp@vt.edu
%     URL: http://etd.vt.edu/latex/
% SAVE AS: etd.tex
% REVISED: September 6, 1997 [GMc 8/30/10]
% 

% Instructions: Remove the data from this document and replace it with your own,
% keeping the style and formatting information intact.  More instructions
% appear on the Web site listed above.

\documentclass[12pt,dvips]{report}

\setlength{\textwidth}{6.5in}
\setlength{\textheight}{8.5in}
\setlength{\evensidemargin}{0in}
\setlength{\oddsidemargin}{0in}
\setlength{\topmargin}{0in}

\setlength{\parindent}{0pt}
\setlength{\parskip}{0.1in}

% Uncomment for double-spaced document.
% \renewcommand{\baselinestretch}{2}

% \usepackage{epsf}

% BHH Packages
\usepackage[]{mcode}
\usepackage[retainorgcmds]{IEEEtrantools}
\usepackage{amsmath}
\usepackage{amssymb}


\begin{document}

\thispagestyle{empty}
\pagenumbering{roman}
\begin{center}

% TITLE
{\large
Simulation of Various Channelizer Structures 
Directed by Cyclostationary Detector}

\vfill

Brian H. Hulette

\vfill

Project Report submitted to the Faculty of the \\
Virginia Polytechnic Institute and State University \\
in partial fulfillment of the requirements for the degree of

\vfill

Master of Engineering \\
in \\
Electrical Engineering

\vfill

Amir I. Zaghloul, Co-Chair \\
Jeffrey H. Reed, Co-Chair \\
T. Charles Clancy

\vfill

February 16, 2015 \\
Falls Church, Virginia

\vfill

Keywords: Metaphysics, Information Retrieval, Spacecraft
\\
Copyright 2015, Brian H. Hulette

\end{center}

\pagebreak

\thispagestyle{empty}
\begin{center}

{\large Simulation of Various Channelizer Structures 
Directed by Cyclostationary Detector}

\vfill

Brian H. Hulette

\vfill

(ABSTRACT)

\vfill

\end{center}

A common problem in Software-Defined Radio (SDR) systems is the problem of
detecting and tuning narrow signals of interest in acquired wideband data. This
involves using some means to detect the frequency of one or more signals of
interest and then tuning, filtering, and decimating them. This problem is
particularly challenging due to the high sample rate of this wideband data,
so efficient algorithms are very desirable.

A MATLAB simulation of an SDR framework for detecting and subband tuning
channelized signals is presented.  Detection is performed using the
spectral-correlation density function [gardner].  These detections are then used to direct
a channelizer which will tune, filter, and decimate all of the detected signals.
Two different channelizer structures are examined - a polyphase
analysis/synthesis channelizer [harris] and an overlap-save filter bank
[Borgerding]. This framework has applications in both Cognitive Radio and
Electronic Warfare.

A summary of relevant background information is provided.  This includes a
discussion of cyclostationarity and the spectral-correlation density function
as well as background on both channelizer structures. For the overlap-save
filter bank it will be shown that in many cases computation can be saved by
re-using the FFT computed for the cyclostationary detector.

In this simulation simple QPSK signals are used to model the signals of 
interest, but the framework is applicable to any modulation or signal type
that has cyclostationary features - which is true of most digital signals. 
While other means of signal detection may be more reliable for specific signals,
a cyclostationary detector's primary strength is its ability to function for many
different digital signals.

\vfill

% GRANT INFORMATION

%That this work received support from the Southeastern Universities
%Research Association (SURA) ``Monticello Library Project'' is purely
%coincidental.

\pagebreak

% Dedication and Acknowledgments are both optional
% \chapter*{Dedication}
\chapter*{Acknowledgments}
I would like to acknowledge the work of both Craig Carlson and Ruth Stoehr. Much of the source for this project for both generating and processing a simple 
QPSK signal came from the ECE5654 project which we completed in the Spring 
of 2013.

I would like to thank my current employer, n~ask, inc., and all of my co-workers
there for their support.

I would also like to thank the members of my committee, particularly Dr.
Zaghloul, for their seemingly endless patience and support.

Finally I would like to thank my friends and family, particularly my girlfriend Meredith, for letting me use ``My Master's" as an excuse for nearly anything.

\tableofcontents
\pagebreak

\listoffigures
\pagebreak

\listoftables
\pagebreak

\pagenumbering{arabic}
\pagestyle{myheadings}

\chapter{Introduction}
\label{sec:intro}
\markright{Brian H. Hulette \hfill Chapter \ref{sec:intro}. Introduction \hfill}

A common problem when designing a Software-Defined Radio (SDR) system is the problem of detecting and tuning signals of interest. A typical SDR will sample at a relatively high rate to acquire a wideband signal. Then it will attempt to detect the channels that contain frequencies of interest, and then tune, filter, decimate, and demodulate those signals. 

%TODO: links to SDR frameworks (webSDR)
%TODO: image of SDR plotting
Many popular SDR applications perform this detection step with a "man-in-the-loop".  A falling raster of the wideband data is presented and the man-in-the-loop will "click-tune" to select  the frequencies that (s)he would like to process. One common version of this is to acquire a wideband slice of the HAM radio spectrum and then click tune to different frequencies to demodulate new FM Push-to-Talk signals and listen to them.

In this paper I will investigate and simulate a structure that automates this process - but rather than working for a single analog signal, this framework is intended for an arbitrary number of digital signals. Detection is performed by a cyclostationary detector rather than by a man-in-the-loop, in order to exploit the cyclostationary properties exhibited by most digital signals with a fixed baud rate.  Tuning, filtering, and decimating is performed by a channelizer, so that many signals can be processed at once.  I will examine two different channelizer structures. They will be compared on two important criteria: 1) Their ability to accurately reproduce every detected signal, and 2) Their computational efficiency when combined with a cyclostationary detector.

In Chapter~\ref{sec:cyclo} I will provide background information on cyclostationary properties and the cyclic spectra upon which my detector relies.  Chapter~\ref{sec:chan} discusses and compares both of the channelizer structures that I will simulate. Chapter~\ref{sec:sim}  discusses the MATLAB simulation I have created and the results of it. Finally, Chapter~\ref{sec:source} provides annotated source code for key parts of the simulation, as well as locations where the entire source code can be found.

\chapter{Cyclostationary Detection}
\label{sec:cyclo}
\markright{Brian H. Hulette \hfill Chapter \ref{sec:cyclo}. Cyclostationary Detection \hfill}

\section{Cyclostationarity}
\label{sec:cyclo_prop}
Often signals are modeled as stochastic random processes with properties such as
mean and variance, but many man-made signals, particularly digital communications,
can be modeled with another second-order statistical property called
\emph{cyclostationarity}. This is simply a property that implies the
auto-correlation function fluctuates at a particular rate\cite{Gardner1}.
% Gardner1 = Exploitation of Spectral Redundancy in Cyclostationary Signals


\section{Exploiting Cyclostationarity for Detection}
\label{sec:exploit_cyc}
For the problem defined in Chapter \ref{sec:intro} we can define the wideband
signal $W(t)$ as the sum of $N$ cyclostationary signals, tuned to
various frequencies, in an AWGN channel:
\begin{IEEEeqnarray*}{lCl}
    W(t) & = & \sum C_{i}(t) e^{j 2 \pi f_i t} + N(t)
\end{IEEEeqnarray*}

Where the $C_i(t)$ are cyclostationary random processes at various baud rates,
tuned to corresponding frequencies $f_i$ and $N(t)$ is the AWGN.

We would like to exploit the cyclostationarity of the $C_i$ waveforms to
identify the frequency, $f_i$, of those waveforms which are transmitted at a
specific baud rate.


\chapter{Channelizer Structures}
\label{sec:chan}
\markright{Brian H. Hulette \hfill Chapter \ref{sec:chan}. Channelizer Structures\hfill}
\section{Polyphase Analysis/Synthesis Channelizer}
\label{sec:poly_chan}

\section{Overlap-save Filter Bank}
\label{sec:filter_bank}

\chapter{Simulation}
\label{sec:sim}
\markright{Brian H. Hulette \hfill Chapter \ref{sec:sim}. Simulation \hfill}

\chapter{Results}
\label{sec:results}
\markright{Brian H. Hulette \hfill Chapter \ref{sec:results}. Results \hfill}

%%%%%%%%%%%%%%%%%
%
% Include an EPS figure with this command:
%   \epsffile{filename.eps}
%

%%%%%%%%%%%%%%%%
%
% Do tables like this:

 \begin{table}
 \caption{The Graduate School wants captions above the tables.}
\begin{center}
 \begin{tabular}{ccc}
 x & 1 & 2 \\ \hline
 1 & 1 & 2 \\
 2 & 2 & 4 \\ \hline
 \end{tabular}
\end{center}
 \end{table}

%%%%%%%%%%%%%%%%%%%%%%%%%%%%%%%%

% If you are using BibTeX, uncomment the following:
% \thebibliography
%
% Otherwise, uncomment the following:
% \chapter*{Bibliography}

% \appendix

% In LaTeX, each appendix is a "chapter"
\chapter{Project Source}
\label{sec:source}
\markright{Brian H. Hulette \hfill Chapter \ref{sec:source}. Project Source \hfill}
% TODO: make sure this is a link - and make the project public!
The full source is hosted on GitHub (\texttt{https://github.com/TheNeuralBit/cyclo\_channelizer}), but I will include some of the critical files below.

\section{\texttt{cyclic\_spectrum.m}}
\lstinputlisting{../cyclic_spectrum.m}
\section{\texttt{synthesis\_channelizer.m}}
\lstinputlisting{../synthesis_channelizer.m}
\section{\texttt{analysis\_channelizer.m}}
\lstinputlisting{../analysis_channelizer.m}

\end{document}
