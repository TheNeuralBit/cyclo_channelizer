%t PROJECT: <ETD> Electronic Thesis and Dissertation Initiative
%   TITLE: LaTeX report template for ETDs in LaTeX
%  AUTHOR: Neill Kipp, nkipp@vt.edu
%     URL: http://etd.vt.edu/latex/
% SAVE AS: etd.tex
% REVISED: September 6, 1997 [GMc 8/30/10]
% 

% Instructions: Remove the data from this document and replace it with your own,
% keeping the style and formatting information intact.  More instructions
% appear on the Web site listed above.

%\documentclass[12pt,dvips]{report}
\documentclass[12pt]{report}

\setlength{\textwidth}{6.5in}
\setlength{\textheight}{8.5in}
\setlength{\evensidemargin}{0in}
\setlength{\oddsidemargin}{0in}
\setlength{\topmargin}{0in}

\setlength{\parindent}{0pt}
\setlength{\parskip}{0.1in}

% Uncomment for double-spaced document.
% \renewcommand{\baselinestretch}{2}

% \usepackage{epsf}

% BHH Packages
\usepackage[]{mcode}
\usepackage[retainorgcmds]{IEEEtrantools}
\usepackage{amsmath}
\usepackage{amssymb}
\usepackage{graphicx,dblfloatfix}
\usepackage{subfig}
\usepackage{hyperref}
\graphicspath{ {./images/} }
\hypersetup{
    colorlinks=true,
    urlcolor=blue,
    citecolor=black,
    linkcolor=black
}
\usepackage[section]{placeins}
% END BHH Packages


\begin{document}

\newcommand{\fourier}{\mathcal{F}}

\thispagestyle{empty}
\pagenumbering{roman}
\begin{center}

% TITLE
{\large
Simulation of Various Channelizer Structures 
Directed by Cyclostationary Detector}

\vfill

Brian H. Hulette

\vfill

Project Report submitted to the Faculty of the \\
Virginia Polytechnic Institute and State University \\
in partial fulfillment of the requirements for the degree of

\vfill

Master of Engineering \\
in \\
Electrical Engineering

\vfill

Amir I. Zaghloul, Co-Chair \\
Jeffrey H. Reed, Co-Chair \\
T. Charles Clancy

\vfill

February 16, 2015 \\
Falls Church, Virginia

\vfill

Keywords: SDR, Cyclostationarity, Detection, Channelizer
\\
Copyright 2015, Brian H. Hulette

\end{center}

\pagebreak

\thispagestyle{empty}
\begin{center}

{\large Simulation of Various Channelizer Structures 
Directed by Cyclostationary Detector}

\vfill

Brian H. Hulette

\vfill

(ABSTRACT)

\vfill

\end{center}

One common problem in Software-Defined Radio (SDR) systems is that of
detecting and then isolating narrow signals of interest in wideband sampled
data. This involves using some means to detect the frequency of one or more
signals of interest and then tuning, filtering, and decimating them. This
problem is particularly challenging due to the high sample rate of this
wideband data, so efficient algorithms are very desirable. Solutions to this
have applications in many areas, including both Cognitive Radio and
Electronic Warfare.

A MATLAB simulation of an SDR framework for detecting and sub-band tuning
channelized signals is presented, with a focus on finding an efficient means to
combine the two algorithms. Detection is performed using the
Spectral Correlation Density (SCD) function which exploits the cyclostationary
property of digital signals \cite{Gardner1}.  These detections are then used to
direct a channelizer which will tune, filter, and decimate all of the detected
signals. Two different channelizer structures are examined: a polyphase
analysis/synthesis channelizer \cite{Harris1} and an overlap-save filter bank
\cite{Borgerding1}.

A summary of relevant background information is provided.  This includes
a discussion of cyclostationarity and the SCD as well as background on both
channelizer structures. For the overlap-save filter bank it will be shown that
in many cases computation can be saved by using the same FFT computation for
both the cyclostationary detector and the channelizer.

In this simulation, simple QPSK signals are used to model the signals of
interest, but the framework is applicable to any modulation or signal type that
has cyclostationary features. While other means of signal detection may be more
reliable for specific signals, a cyclostationary detector's primary strength is
its ability to function for a wide variety of digital signals.


\vfill

% GRANT INFORMATION

%That this work received support from the Southeastern Universities
%Research Association (SURA) ``Monticello Library Project'' is purely
%coincidental.

\pagebreak

% Dedication and Acknowledgments are both optional
% \chapter*{Dedication}
\chapter*{Acknowledgments}
First I must acknowledge the work of both Craig Carlson and Ruth Stoehr.
The core of my testbed for this project is a robust QPSK Transmitter/Receiver
pair we designed together for an ECE5654 project in the Spring of 2013.

I would also like to thank the members of my committee, particularly Dr.
Zaghloul, for their patience and support.

Finally, I would like to thank my friends and family, especially my girlfriend
Meredith, for helping me through this years long process, and letting me use
``My Masters" as an excuse for nearly anything.

\tableofcontents
\pagebreak

\listoffigures
\pagebreak

%\listoftables
%\pagebreak

\pagenumbering{arabic}
\pagestyle{myheadings}

\chapter{Introduction}
\label{sec:intro}
\markright{Brian H. Hulette \hfill Chapter \ref{sec:intro}. Introduction \hfill}

A common problem when designing a Software-Defined Radio (SDR) system is the
problem of detecting and tuning signals of interest. A typical SDR will sample
at a relatively high rate to acquire a wideband signal. Then it will attempt to
detect the channels within the acquired frequency range that contain signals of
interest, and then tune, filter, decimate, and demodulate those signals. 

Many popular SDR applications perform this detection step with a ``man in the
loop".  A falling raster of the wideband data is presented and the man in the
loop will select  the frequencies that he would like to process (\cite{WebSDR}
is a web-based implementation of this).  A common example of this is to acquire
a wideband slice of the HAM radio spectrum and then select different
frequencies to demodulate new FM push-to-talk signals and listen to them.

This paper investigates a structure that automates this process - but rather
than processing a single analog signal, this framework is intended to process
an arbitrary number of digital signals at once. Detection is performed by
a cyclostationary detector rather than by a man in the loop. This exploits the
cyclostationary properties exhibited by most digital signals with a fixed
symbol rate.  Tuning, filtering, and decimating is performed by a channelizer,
so that many signals can be processed at once.  Two different channelizer
structures are examined: a polyphase channelizer composed of an analysis
channelizer followed by several synthesis channelizer, and an overlap-save
filter bank. These structures are evaluated on two important criteria:
1) Their ability to accurately reproduce every detected signal, and 2) Their
computational efficiency when combined with a cyclostationary detector.
% TODO: list papers that discuss SCD for various protocols

Chapter~\ref{sec:background} provides background information on cyclostationary
properties and the SCD upon which the detector relies, as well as  both of the
channelizer structures that are simulated. Chapter~\ref{sec:sim}  discusses the
MATLAB simulation that was created and the results of it. Finally,
Appendix~\ref{sec:source} provides annotated source code for key parts of the
simulation, as well as locations where the entire source code can be found.

\chapter{Background Information}
\label{sec:background}
\markright{Brian H. Hulette \hfill Chapter \ref{sec:background}. Background Information \hfill}

\section{Cyclostationary Detection}
\label{sec:cyclo}

\subsection{Cyclostationarity}
\label{sec:cyclo_prop}
Often signals are modeled as stochastic random processes with properties such as
mean and variance, but many man-made signals, particularly digital communications,
can be modeled with another statistical property called
\emph{cyclostationarity}, which implies the signal has some parameter which
varies periodically with time. The frequency of this variation is called the cyclic
frequency, $\alpha$. \cite{Gardner1}.


Of particular interest for this project is second-order cyclostationarity,
where a signal has a periodic auto-correlation, $R_{xx}(t, t+\tau)$:

\begin{IEEEeqnarray}{lCl}
    R_{xx}(t, t+\tau) = E\{x(t)x^*(t+\tau)\} = \sum_{\alpha} R_{xx}^{\alpha}(\tau)e^{j2 \pi \alpha t}
\end{IEEEeqnarray}

The function $R_{xx}^{\alpha}(\tau)$ is the cyclic auto-correlation function (CAF), given by:

\begin{IEEEeqnarray}{lCl}
    R_{xx}^{\alpha}(\tau) = \lim_{T \to \infty} \frac{1}{T}\int_{-T/2}^{T/2} R_{xx}(t, t+\tau)e^{-j2\pi \alpha t} dt
\end{IEEEeqnarray}
The CAF is a function of two parameters, the cyclic frequency, $\alpha$ and the
delay, $\tau$. This function can be used for signal detection in the time
domain. An estimate of the two-dimensional function versus $\alpha$ and $\tau$
is computed, then features within this plane are used for detection (e.g. \cite{Jiandong1, Oner1}).
However, what is of interest for this project is detection in the frequency
domain. This can be performed using the Fourier transform of the CAF, called
the Spectral Correlation Density (SCD), given by:

\begin{IEEEeqnarray}{lCl}
    S_{xx}(\alpha, f) = \int_{-\infty}^{\infty} R_{xx}^{\alpha}(\tau)e^{-j2\pi f \tau} d\tau
\end{IEEEeqnarray}

The SCD is a generalization of the conventional Power-Spectral Density (PSD),
which it reduces to at $\alpha=0$ \cite{Oner1}. Like the CAF, the SCD also
contains unique features based on the modulation type, symbol rate, and carrier
frequency of the transmitted signal(s), which we can exploit for detection.
In \cite{Gardner2}, Gardner defines this SCD for various basic modulation
types, including QPSK, which is used in the presented simulation.

%\section{Exploiting Cyclostationarity for Detection}
%\label{sec:exploit_cyc}
%For the problem defined in Chapter \ref{sec:intro} we can define the wideband
%signal $W(t)$ as the sum of $N$ cyclostationary signals, tuned to
%various frequencies, in an AWGN channel:
%\begin{IEEEeqnarray*}{lCl}
%    W(t) & = & \sum C_{i}(t) e^{j 2 \pi f_i t} + N(t)
%\end{IEEEeqnarray}
%
%Where the $C_i(t)$ are cyclostationary random processes at various symbol rates,
%tuned to corresponding frequencies $f_i$ and $N(t)$ is the AWGN.

%
%We would like to exploit the cyclostationarity of the $C_i$ waveforms to
%identify the frequency, $f_i$, of those waveforms which are transmitted at a
%specific symbol rate.

\subsection{Estimating the SCD}
\label{sec:estimating_scd}
% Use Gardner1 to show that SCD is conjugate multiplication of frequency shifted spectra
% TODO:
% - cyclic wiener relation on Gardner1 pg. 153
% - attempts to derive in lab notebook pg. 20-21
% or just cite Gardner1?
We can estimate a slice of the SCD at cyclic frequency $\alpha$ as
the cross spectral density of the two frequency-shifted time sequences
$x_L(t) = x(t)e^{j\pi\alpha t}$ and $x_U(t) = x(t)e^{-j\pi\alpha t}$ \cite{Gardner1}.

\begin{IEEEeqnarray}{lCl}
    S_{xx}(\alpha, f) & = & X_L(f)X_U^* (f)
\end{IEEEeqnarray}

Note that $x_L(t)$ has been shifted down in frequency by $\alpha/2$, and
$x_U(t)$ has been shifted up by the same amount. In order to estimate the SCD
in this way we must perform two FFTs: one for both $X_L(f)$ and $X_U(f)$.
However, it is possible to estimate both of these spectra using a single FFT,
$X(f)$, in some cases:

\begin{IEEEeqnarray}{lCl}
    S_{xx}(\alpha, f) & = & X_L(f)X_U^* (f) \\
                      & = & X(f - \alpha/2)X^*(f + \alpha/2)
\end{IEEEeqnarray}

Using the second relation we can compute $S_{xx}(\alpha, f)$ using $X(f)$, by
shifting that FFT in either direction by $\alpha/2$, and then conjugate
multiplying the two shifted spectra together. This is significant, since
a simple forward FFT of the acquired data $X(f)$ is useful for other
operations, such as channelization, as we will see in 
Section~\ref{sec:os_filter_bank}.

In order to perform this shift accurately using a single FFT we need to
circular shift the FFT by an integer number of bins.  This means we can only
accurately perform this frequency shift when $\alpha/2$ is a multiple of the
FFT resolution.  Thus $\alpha$ must satisfy the following relationship:
\begin{IEEEeqnarray}{lCl}
    \alpha = \frac{2kf_s}{N_{fft}} \text{, } k \in \mathbb{Z}
\end{IEEEeqnarray}
\label{eq:cyclo_freqs}

If an estimate of the SCD at a cyclic frequency that does not satisfy this
relation is required, then we must take the original approach (perform the
frequency shift in the time domain) to be perfectly accurate. Alternatively, we
can still use a single forward FFT and approximate the shift by interpolating
between bins, but this is only an approximation.


\section{Polyphase Analysis/Synthesis Channelizer}
\label{sec:poly_chan}
The first channelizer structure we will examine relies on the Analysis and
Synthesis Polyphase Channelizers described by Fred Harris \cite{Harris1}.
A Polyphase Analysis channelizer can be used to break up a wideband signal into
$D$ distinct channels, each decimated by a factor of $D$ from the wideband
sample rate.  While a Polyphase Synthesis Channelizer performs the reverse
operation, combining $D$ channels into one wideband signal. The channel
frequencies used in both structures are integer multiples of the output sample
rate:

\begin{IEEEeqnarray}{lCl}
    f_k = k \frac{f_s}{D}
\end{IEEEeqnarray}

where $f_k$ is the center frequency of channel $k$ and $f_s$ is the input sample
rate. A Polyphase Synthesis Channelizer performs the reverse operation - it
combines $D$ distinct channels with equal sample rate into a single wideband
signal.

These two structures can be combined to create a very flexible channelizer that
can tune channels with arbitrary center frequencies and various bandwidths
\cite{Harris2} - a simplified version of this structure is discussed in
Section~\ref{sec:combine_analysis_synthesis}.

In \cite{Harris1}, Harris shows how the Polyphase Analysis Channelizer works by
starting with a basic filter/decimator and making incremental modifications
until it becomes an Analysis Channelizer as shown in Figure
\ref{fig:polyphase_analysis}. A brief summary of this description is provided here, which follows Figure~\ref{fig:polyphase_proof}.

\begin{figure}[h!]
\centerline{
    \subfloat[Single channel of a simple tuning, filtering, and decimating channelizer. $h(n)$ is a baseband low-pass filter.]
    {\includegraphics[width=0.45\textwidth]{polyphase_1}%
    \label{fig:polyphase_1}}
    \hfill
    \subfloat[Equivalency theorem allows the tuning step to be moved after the decimation. The baseband filter must be shifted up to $f_k$. If $f_k$ is a multiple of the output sample rate,  $\frac{f_s}{D}$, the tuning step can be dropped.]
    {\includegraphics[width=0.45\textwidth]{polyphase_2}%
    \label{fig:polyphase_2}}
}
\centerline{
    \subfloat[The Noble Identity allows the filtering and decimation to be combined to remove excess computations.]
    {\includegraphics[width=0.45\textwidth]{polyphase_3}%
    \label{fig:polyphase_3}}
    \hfill
    \subfloat[The series of delays and decimation steps can be replaced with an input commutator and complex phasors can be added after each filter to select channel $k$.]
    {\includegraphics[width=0.45\textwidth]{polyphase_4}%
    \label{fig:polyphase_4}}
}
\caption{Creating a single channel of a polyphase analysis channelizer}
\label{fig:polyphase_proof}
\end{figure}

We start with a single channel of a basic channelizer in Figure
\ref{fig:polyphase_1}. This channel consists of a multiplication with a complex
phasor to tune the desired center frequency down to baseband, followed by
a low-pass filter, represented by $h(n)$, and finally a decimation by $D$. The
first modification we can make takes advantage of the Equivalency Theorem,
which says that we can switch the order of the tuner and the filter, \textbf{if}
the filter is changed to a bandpass filter centered at $f_k$. This modification
is shown in Figure \ref{fig:polyphase_2}. In this figure the tuner has also
been moved after the decimator. Note that if the channel frequency, $f_k$, is
an integer multiple of the output sample rate, $\frac{f_s}{D}$, then the tuner
simplifies to $e^{j2\pi n} = 1$, so it can be dropped completely.  Thus we will
restrict the polyphase analysis channelizer to center frequencies which are
integer multiples of the output sample rate.

Next we can use the Noble Identity to switch the place of the decimation and
the filter, as shown in Figure \ref{fig:polyphase_3}. In order to do so we need to define a set of new filters, $H_r(z)$, such that:

\begin{IEEEeqnarray}{lCl}
    H(z) = H_0(z^D) + z^{-1}H_1(z^D) + \hdots + z^{-(D-1)}H_{D-1}(z^D)
\end{IEEEeqnarray}

This means that each filter, $H_r(z)$, has has an impulse response which is
$h(n)$ shifted by $r$ samples and decimated by $D$. Using the Noble Identity in
this way saves us from performing computations for samples that will just be
dropped by the decimator. Note that in Figure \ref{fig:polyphase_3} the tuner
has been dropped as well, since we are restricting the channelizer to
frequencies which are multiples of the output sample rate.

Finally, we complete the structure for a single channel of a polyphase analysis
channelizer in Figure \ref{fig:polyphase_4}. The combination of delays and
decimators at the front of the previous structure is actually just
a commutator. In \cite{Harris1} Harris explains this by thinking about the
decimators as a switch that closes every $D$ samples. So in Figure
\ref{fig:polyphase_3}, when all of the switches close the filter at the bottom
gets the oldest sample, and each filter as you go up gets one newer sample
until you get to the most recent sample on the top. The next sample is not
processed until the switches close again, and it is passed to the filter at the
bottom. With this explanation it is easy to see the whole structure can be
replaced with a commutator.

The final structure has one more modification. The outputs are multiplied by
complex phasors which select the individual channel, $k$, centered at $f_k$.
A detailed description of this can be found in \cite{Harris1}. When all of the
channels are combined to form the full channelizer these complex phasors all
represent a DFT.  The set of phasors that are multiplied and then summed
together for channel $k$ correspond to the $k$th output of a DFT:

\begin{IEEEeqnarray}{lCl}
    y_k(n) & = & \sum_{r=0}^{D-1} y_r(n) e^{j(2\pi/D)rk} 
\end{IEEEeqnarray}

Where $y_r(n)$ represents the output of filter $H_r(z)$, and $y_k(n)$ is the
output of channel $k$. Thus we can compute all $D$ of the complex phasors with
a $D$ point FFT, as shown in Figure \ref{fig:polyphase_analysis}. Figure
\ref{fig:polyphase_synthesis} shows the complementary structure: the Polyphase
Synthesis Channelizer. It is essentially a mirror image of the Analysis case
- an IFFT followed by filters and a commutator. It combines $D$
channels at an equal sample rate, $f_s$, into a single wideband signal at
sample rate $Df_s$. The following section shows how these two structures can be
used together to produce a very flexible channelizer.

\begin{figure}[h!]
\centerline{
    \subfloat[Analysis]
    {\includegraphics[height=4cm]{polyphase_analysis}%
    \label{fig:polyphase_analysis}}
    \hfill
    \subfloat[Synthesis]
    {\includegraphics[height=4cm]{polyphase_synthesis}%
    \label{fig:polyphase_synthesis}}
}
\caption{Full Polyphase Analysis/Synthesis Channelizer Structures}
\label{fig:poly_analysis_synthesis_structs}
\end{figure}

\subsection{Combining Analysis and Synthesis Channelizers}
\label{sec:combine_analysis_synthesis}
The primary limitations of an Analysis Channelizer alone are that it is limited
to a single output sample rate, and the channel center frequencies must be
a multiple of that output sample rate. These limitations are perfectly
acceptable when every signal being processed is at the same bandwidth and
symbol rate, and they are channelized with a known spacing, but in many SDR
applications this is not the case. One solution to this problem is to use
analysis and synthesis structures \emph{together} to create one very flexible
structure.

Figure \ref{fig:analysis_and_synthesis} illustrates one simple way that this
might be implemented. The first step of this process uses an Analysis
Channelizer to break up the wideband into small equal parts (1). Then,
Synthesis Channelizers are used to re-combine the parts of the signal that
correspond to signals of interest (2). Finally, complex phasors are used to
remove the frequency offset created by the Synthesis Channelizers, as well as
any residual offset based on the desired center frequency (3).

\begin{figure}[h!]
    \begin{center}
    \includegraphics[width=0.8\textwidth]{polyphase}%
    \end{center}
    \caption{
Using Polyphase Analysis/Synthesis Channelizer Structures in concert. Step (1)
is an Analysis Channelizer that breaks the signal into 16 channels, Step (2) is
a set of Synthesis Channelizers for each signal of interest, and Step (3) is
a complex phasor that corrects the frequency offset. Note that this figure
depicts the signal in the frequency domain, simply for ease of illustration -
all inputs and outputs are time-domain signals.
    }
    \label{fig:analysis_and_synthesis}
\end{figure}

The analysis channelizer can be running all the time, while the synthesis
channelizers and phasors in steps (2) and (3) can be dynamically allocated as
signals of interest are detected (by some external detector).

\cite{Harris2} describes a more complex take on this same concept, but this
simplified version is sufficient for this project.
% TODO: more detail about Harris2



\subsection{Limitations}
\label{sec:poly_limitations}
We have already examined the limitations of a Polyphase Analysis
Channelizer used alone: every channel must be at the same output sample
rate, $f_s/D$, and they must have center frequencies which are multiples of
that sample rate.

If the combination analysis/synthesis structure is used then this limitation
can be avoided, however there are still other issues. First, the output sample
rate must still be a decimation of the input sample rate - arbitrary rates are
not allowed. Second, while the output signals can be reconstructed quite well
after they have been split by the analysis channelizer, the reconstruction is
not perfect and could introduce errors. Harris discusses the concept of
a ``perfect reconstruction" filter \cite{Harris2} which may circumvent this
limitation, but it is beyond the scope of this project.

Finally, there is no obvious way that this structure can be efficiently
combined with SCD estimation for detection. Nothing about an analysis
channelizer can be re-used for the detection approach used in this project.

\subsection{Advantages}
\label{sec:poly_advantages}
The primary advantage of a lone Polyphase Analysis Channelizer is its
simplicity and efficiency, but its limitations make it an untenable solution
for this project.

The combination Polyphase Analysis/Synthesis Channelizer adds some complexity,
but it is still quite efficient for the amount of flexibility that it provides.
Another major benefit of both Analysis and Synthesis Channelizers is that they
can be implemented in hardware relatively easily. That means the Combination
Analysis/Synthesis structure could be implemented in hardware. One can imagine
a hardware implementation of this structure where an Analysis Channelizer is
always running, and a bank of Synthesis Channelizers are dynamically connected
and disconnected as signals
are detected.

% TODO: this needs to be expanded

\section{Overlap-save Filter Bank}
\label{sec:os_filter_bank}
The Overlap-Save Filter Bank structure used in this simulation is based
entirely on a description by Mark Borgerding from March 2006
\cite{Borgerding1}.  Borgerding's concept is based on the well known
Overlap-Save (OS) fast convolution technique.

OS fast convolution can be used to speed up convolution with
a filter that has a long impulse response. The concept is that rather than
convolving in the time-domain, $O(N^2)$, it is faster to first
perform an FFT of both the signal and the filter and conjugate multiply in the
frequency domain, then IFFT to go back to the time domain, $O(N\log_2N)$.
There is nothing new about this idea, but Borgerding's innovation is that he
shows how to extend this concept to tune, filter and decimate any number of
channels with arbitrary center frequencies and bandwidths.

We will now see how the OS Filter Bank performs Tuning, Filtering, and
Decimation. The same terms that Borgerding defines will be used in this
description:

\begin{center}
\begin{tabular}{ll}
    $x(n)$        & Input data \\
    $h(n)$        & Baseband filter response \\
    $y(n)$        & Tuned, filtered and decimated output data \\
    $P$           & Length of $h(n)$ \\
    $N$           & FFT size \\
    $D$           & Decimation factor \\
    $V = N/(P-1)$ & Ratio of FFT size to filter order \\
\end{tabular}
\end{center}

\emph{Tuning:} If the frequency of the signal of interest corresponds to one of
the FFT bins then we can simply circular shift the FFT output to place that bin
at the center. Then filtering can be performed at baseband.  If all channels
satisfy this criterion then we can re-use a single forward FFT for every
channel, simply by circular shifting it to the appropriate bin in every case. 
This approach is shown in Figure \ref{fig:overlap_save_shift}.

However, if this condition is not satisfied then we will need to perform the
frequency shift in the time domain by multiplying by a complex phasor.
Fortunately, there is a still a way to re-use the forward FFT in this case.
After taking an FFT of the non frequency-shifted signal we can shift the
baseband filter response up to the desired frequency, then after the filtered
signal is returned to the time domain with an IFFT, we perform the frequency
shift with a complex phasor.  In addition to allowing us to re-use the forward
FFT for each channel, this is more efficient than frequency shifting before the
forward FFT since the multiplication is performed after decimation. This
approach is shown in Figure \ref{fig:overlap_save_time_domain}.

\begin{figure}[bh!]
\centerline{
    \subfloat[Performing frequency shift by circular shifting the FFT output]
    {\includegraphics[width=0.48\textwidth]{overlap_save_shift}%
    \label{fig:overlap_save_shift}}
    \hfill
    \subfloat[Performing precise frequency shift in the time-domain]
    {\includegraphics[width=0.48\textwidth]{overlap_save_time_domain}%
    \label{fig:overlap_save_time_domain}}
}
\caption{Overlap-Save Filter Bank Structures}
\label{fig:overlap_save_filter_banks}
\end{figure}

\emph{Filtering:} As discussed earlier, this structure relies on OS fast
convolution for filtering. The low-pass filter impulse response for each
channel is Fourier Transformed and then conjugate multiplied with the FFT of
the input data after the signal of interest has been tuned to baseband.
Alternatively, if tuning is being performed in the time domain, the filter
impulse response must be tuned up to the signal frequency before being Fourier
transformed.

\emph{Decimation:} Following the frequency domain filtering we have two
different options for decimation. The first option is to perform a full size
inverse FFT of the output, and then decimate in the time domain. The issue with
this approach is that we are computing a larger inverse FFT than we really need
to. Borgerding suggests decimating in the frequency domain as a solution to this.
The approach (called ``Wrap/Sum" in Figure
\ref{fig:overlap_save_filter_banks}) is simple: coherently add together the
components of the frequency spectrum that would be aliased upon time domain
decimation. For example, if the FFT size is 1024 and we need to decimate by
a factor of 4, then coherently sum FFT samples 0 to 255, 256 to 511, 512 to 
767 and 768 to 1024.  The result is a 256 sample frequency
spectrum which we can now inverse FFT to produce the decimated output. The
only trick with this approach is that we must discard just $(P-1)/D$
samples rather than $P-1$ to account for the overlap.  This reveals one
restriction of this structure: the filter order $P-1$ must be an integer
multiple of the decimation, $D$.

\subsection{Limitations}
\label{sec:os_limitations}
There are a few limitations of this structure that are worth mentioning. First,
as mentioned earlier, the filter order $P-1$ must be an integer multiple of the
decimation factor $D$. This problem is pretty easy to solve though - simply
zero-pad the filter to achieve an appropriate length. Another limitation that
is potentially challenging is that the FFT size, $N$, must be an integer
multiple of the decimation rate, $D$. So decimation rates whose prime factors
are larger than 2 or 5 (or others, depending on the FFT implementation) could
lead to FFT inefficiency for this structure.

One more limitation occurs when attempting to rotate the FFT to frequency
shift.  As previously mentioned, the precision of the frequency shift is
limited by the resolution of the FFT. However, the precision is also limited
further by $V$, the ratio of FFT size to filter order. This is because we must
restrict mixing to the frequencies whose period completes in $L=N-(P-1)$
samples. Borgerding provides the following equation for computing the number of
FFT bins to rotate to shift to frequency $f$ (\cite{Borgerding1} Equation (1)):

\begin{equation}
    N_{rot} = \text{round}\left( \frac{Nf}{Vf_s} \right) V
\end{equation}

Where $f_s$ is the wideband sample rate. This equation simply adjusts $N_{rot}$
to the nearest multiple of $V$ bins. It is worth noting that the solution
presented for making $P-1$ an integer multiple of $D$ is only making this
problem worse by making $V$ larger - but there's nothing to be done about that
other than opting for a shorter filter.

\subsection{Advantages}
\label{sec:os_advantages}

The greatest benefit of the Overlap-Save Filter Bank for this project is the
ease with which it can be combined with SCD Estimation for cyclostationary
detection. Since the first step for both algorithms is a forward FFT of the
wideband input, if we were to design a joint cyclostationary
detector/Overlap-Save Filter Bank we could re-use the same forward FFT for both
algorithms.

The only design challenge is finding a combination of FFT size, sample rate,
and decimation factor that will work for both algorithms.

% TODO: Analysis of this problem. Decimation factor is related to alpha by the amount of oversampling.
% Started this analysis a bit in lab notebook pg. 23

%We can say that the output sample rate is some integer multiple of the symbol rate:
%\begin{equation}
%    f_s' = \beta \alpha \text{, } \beta = 1,2,\hdots
%\end{equation}
%Where the integer multiple $\beta$ is simply the amount of oversampling. Then we can 

Additionally, averaging the overlapped FFT frames together may have an effect
on the accuracy of the SCD estimation, but analyzing this effect is beyond the
scope of this project.

\chapter{Simulation}
\label{sec:sim}
\markright{Brian H. Hulette \hfill Chapter \ref{sec:sim}. Simulation \hfill}
A MATLAB simulation of these structures has been written. Directions to find
all of the source code can be found in Appendix \ref{sec:source}. Each module
- testbed, cyclostationary detector, polyphase analysis/synthesis channelizer,
and overlap-save filter bank - was developed and tested separately. The
detector was then combined with each channelizer and their performance was
evaluated.  The same logical flow is followed in this Chapter as the modules
are presented.

\section{Testbed}
Each Module is tested with the same wideband signal, shown in
Figure~\ref{fig:test_signal}. It is a waveform sampled at 10 MHz with three
QPSK signals at different symbol rates and center frequencies:
\begin{itemize}
    \item{312.5 ksymbols/s at -2.5 MHz}
    \item{156.25 ksymbols/s at 0 MHz}
    \item{625 ksymbols/s at 2.5 MHz}
\end{itemize}
Note that the symbol rates were all chosen as fractions of the wideband sample
rate by design, as this allows the approximation discussed in
Section~\ref{sec:estimating_scd} to be used.

\begin{figure}[h]
    \begin{center}
        \includegraphics[width=0.95\textwidth]{test_signal}
    \end{center}
    \caption{Frequency spectrum of the 10 MHz test signal used for each module}
    \label{fig:test_signal}
\end{figure}

A simulation of a QPSK transmitter/receiver pair is used to generate the
signals used in this signal, and to attempt to demodulate the output of the
channelizers. This transmitter channel encodes the data with a convolutional
code, packetizes the encoded data, and appends a synchronization sequence to
each packet. The receiver uses a squared spectrum to correct any major
frequency offsets, then uses the synchronization sequence to correct timing and
phase offsets. on a per packet basis. Finally, it performs channel decoding and
outputs the demodulated bits\footnote{This transmitter/receiver pair is based
on a project the author completed for ECE5654 along with Craig Carlson and Ruth Stoehr.
The original report for this project is available at
\url{https://github.com/TheNeuralBit/cyclo_channelizer/raw/master/reference/ECE5654_Project_Report.pdf}.}.

\section{Cyclostationary Detector}
\label{sec:sim_cyclo}
The next module is the cyclostationary detector. It works by computing
estimates of the SCD at particular cyclic frequencies and searching for
features in those estimates. The first step, estimating the SCD, can be
performed by the MATLAB function \texttt{cyclic\_spectrum(...)}, which will
compute an estimate of the given data at a specific cyclic frequency, $\alpha$.
It accepts a few additional parameters:

\begin{description}
    \item[Mode:] Set to either of the techniques described in Section
    \ref{sec:estimating_scd}: Exact time-domain frequency shifts, or a single
    FFT with frequency-domain shifts. The latter is what we are primarily concerned
    with, for its efficiency.
    \item[FFT Size:] Size of the FFT used to go to the frequency domain.
    \item[Averaging:] Will average $N$ estimates together to produce a more
        accurate
    result.
\end{description}

According to \cite{Gardner2}, QPSK signals generate a large peak in the SCD at
the signal's center frequency when $\alpha$ is equal to the signal baud rate.
So for this application estimates are generated at cyclic frequencies
corresponding to baud rates of interest. Figure \ref{fig:cyclo_estimates} shows
examples of these estimates for our test signal.

\begin{figure}[bh!]
\centerline{
    \subfloat[SCD at $\alpha = 0$, Equivalent to a PSD]
    {\includegraphics[width=0.45\textwidth]{cyclo_0}%
    \label{fig:cyclo_0}}
    \hfill
    \subfloat[SCD at $\alpha = 156.25$ kHz]
    {\includegraphics[width=0.45\textwidth]{cyclo_156250}%
    \label{fig:cyclo_156250}}
}
\centerline{
    \subfloat[SCD at $\alpha = 312.5$ kHz]
    {\includegraphics[width=0.45\textwidth]{cyclo_312500}%
    \label{fig:cyclo_312500}}
    \hfill
    \subfloat[SCD at $\alpha = 625$ kHz]
    {\includegraphics[width=0.45\textwidth]{cyclo_625000}%
    \label{fig:cyclo_625000}}
}
\caption{Estimates of the SCD at various cyclic frequencies. FFT size = 1024, Averaged over 10 FFTs.}
\label{fig:cyclo_estimates}
\end{figure}

In each of these estimates it easy to detect the large, narrow peak at the
center frequency of the signal with the corresponding baud rate. However, there
are also other features, which agree with the theoretical SCD from
\cite{Gardner2}.  The higher baud rate signals generate wide peaks in the lower
cyclic frequency estimates, like the peak at 2.5 MHz in Figure
\ref{fig:cyclo_156250}. These are easy to distinguish from the main peak with
the human eye since they are much wider, but distinguishing them computationally
is more challenging. The lower baud rate signals also generate features in the
higher cyclic frequency estimates, in the form of two, much smaller peaks,
straddling the actual center frequency. Examples of this are visible at 0 MHz
in Figure \ref{fig:cyclo_312500} and \ref{fig:cyclo_625000}. These features are
easily filtered out by either a human or a computer with a simple threshold.

The second portion of the cyclostationary detection must use these SCD estimates to 
generate a list of detected frequencies, and the corresponding signal's
baud rate. In order to do so the following algorithm has been devised. It accepts the input wideband data, and a list of potential baud rates that are of interest.

\begin{enumerate}
    \item Initialize two empty lists \texttt{f[]} (center frequencies) and \texttt{b[]} (corresponding baud rate for each frequency)
    \item Sort list of potential baud rates from lowest to highest
    \item Estimate SCD at $\alpha$ equal to the first (lowest) baud rate
    \label{loop_ref}
    \item Detect peaks in the SCD. Use a threshold and a minimum peak separation
        to filter out false alarms.
    \item For each detected peak, check if an entry already exists in
        \texttt{f[]} near that frequency. If so, replace it with this peak's
        frequency and replace the corresponding entry in \texttt{b[]} with
        $\alpha$. Otherwise, append a new entry to \texttt{f[]} and
        \texttt{b[]} with the same values.
    \item Repeat from \ref{loop_ref} using the next baud rate in the list
\end{enumerate}

This is a very simple algorithm, but it works well for this test. It is
fully implemented in the \texttt{cyclostationary\_detect(...)} function. This approach will essentially select the highest symbol rate that appears to be active at a given center frequency.

In an actual application a more robust approach should be used. One possibility
which could be investigated but is beyond the scope of this project would be to
use a morphological filter to isolate only very narrow peaks. Between that and
the thresholding, other features would be easily ignored. For this project the
simple approach was taken, because the primary focus is on efficiently
estimating the SCD.

There is a major limitation to the implementation of this approach which is
important to note.  The SCD estimates are generated only based on the beginning
of the input signal. This means that signals are detected based on only the
first $N_{FFT}N_{Avg}$ samples. This works well in this application
since the test signals are on at the beginning of time and continue until there
is no more data to transmit, but in reality signals are going up and down
all the time. In an actual application the cyclostationary detector should be
run continuously to detect when signals come up and go down.

\section{Polyphase Analysis/Synthesis Channelizer}
\label{sec:sim_poly}
The simulation of the polyphase analysis channelizer is broken up into three
major parts: \texttt{analysis\_channelizer(...)},
\texttt{synthesis\_channelizer(...)}, and \texttt{polyphase\_channelizer(...)}.  The
latter of these combines the first two to implement the flexible channelizer
described in Section~\ref{sec:combine_analysis_synthesis}.

The implementation of \texttt{analysis\_channelizer(...)} accepts time domain
input data and a number of channels to produce, and outputs that number of
channels in the time domain. No additional configuration is necessary.
A low-pass filter with cut-off at the output's nyquist frequency is designed and
used. The output is indexed such that the the frequency of the channel at index
$k$ is given by:
%.  This indexing is illustrated in
%Figure~\ref{fig:polyphase_channel_indexing}.
\begin{IEEEeqnarray}{lCl}
    f_k = -\frac{f_s}{2} + k\frac{f_s}{D} \text{, } k=1\hdots{}D
\end{IEEEeqnarray}
Where k follows the standard indexing for MATLAB arrays.

% TODO create this figure

\texttt{synthesis\_channelizer(...)} is designed to perfectly complement
\texttt{analysis\_channelizer(...)}, so that if it were passed the list of
channelized data produced by the synthesis channelizer as an argument, it would
produce a replica of the original input. Internally, it designs and uses
a low-pass filter with cutoff at the input's nyquist frequency, so that an
analysis and synthesis channelizer of the same size will use the same filter.

Finally, \texttt{polyphase\_channelizer(...)} utilizes both of these functions to
implement a flexible channelizer structure.  It accepts the input data, its
sample rate, and two additional configuration parameters:
\begin{description}
    \item[Frequencies:] A list of the center frequencies of the signals of interest
    \item[Decimations:] A list of the decimation factors to be used for each
        corresponding center frequency.  Every decimation factor must be
        a factor of the highest decimation factor.
\end{description}
The simulation will first create a synthesis channelizer with decimation equal
to the highest requested decimation (Step 1 from
Figure~\ref{fig:analysis_and_synthesis}), an example of the output of this step
for our test signal is shown in Figure~\ref{fig:polyphase_splits}.  For all
requested outputs with this decimation factor, the nearest output of the
analysis channelizer will be selected and frequency shifted so that the
desired center frequency is at baseband, and passed through. For every
remaining output, a synthesis channelizer will be produced to combine the
channels surrounding that frequency (Step 2) and a complex phasor will shift
the desired center frequency down to baseband (Step 3).

\begin{figure}[ht!]
    \includegraphics[width=\textwidth]{polyphase_splits}%
    \caption{Output of a 16 channel polyphase analysis channelizer operating on
             the test signal. X Axes are frequency in Hz, Y Axes are magnitude
             in dB.}
    \label{fig:polyphase_splits}
\end{figure}

The major limitation of this implementation is that certain frequencies at the
highest decimation factor \emph{cannot} be produced. Any signal at the lowest
output sample rate with center frequency at the boundary between two outputs of
the analysis channelizer will not be channelized correctly. A solution to this
would be to force the analysis channelizer to use a higher decimation factor
than any of the requested outputs.  However there are certain trade-offs here:
it increases the computational complexity, since a narrower filter is
required, and every output is separated into more parts before being
combined which could harm the signal integrity.

\subsection{Adding Cyclostationary Detection}
\label{sec:sim_poly_cyclo}
As discussed in Section~\ref{sec:poly_limitations}, there is no way to
efficiently combine this polyphase channelizer structure with wideband SCD
estimation.  Thus, this simulation simply uses an unmodified version of the
cyclostationary detector discussed previously to direct an unmodified polyphase
channelizer.

This function, \texttt{cyclostationary\_and\_polyphase(...)} accepts the input
data and performs detection and channelization. The detector can be configured
with all the same parameters discussed in Section~\ref{sec:sim_cyclo}, and
outputs a list of center frequencies and corresponding symbol rates.
The symbol rates are converted to decimation factors based on the desired number
of samples per symbol, specified by the user. The list of frequencies and
corresponding decimation factors are then passed to the channelizer.

\begin{figure}[bh!]
    \includegraphics[width=\textwidth]{cyclo_poly_results}%
\caption{Output of the combined cyclostationary detector and polyphase channelizer. All three test signals sampled at four samples per symbol.}
\label{fig:cyclo_poly_results}
\end{figure}

Figure~\ref{fig:cyclo_poly_results} shows the output of this simulation when
run on our test signal. In this case the channelizer was configured to output
each signal at four samples per symbol. Note the small peaks in the $2.5$ MHz
signal around $\pm1$ MHz - these are not present in the original signal, they
are produced from aliasing in the analysis channelizer, which can be seen in
the corresponding outputs in Figure~\ref{fig:polyphase_splits}. Regardless, the
high SNR of these outputs allowed every channel to be demodulated with a 0\%
BER.
% 0.9375 MHz, to be exact

\section{Overlap-Save Filter Bank}
\label{sec:sim_os}
The final module is the Overlap-Save Filter Bank. The Filter Bank is
implemented with tuning performed in the frequency domain by circular shifting
the FFT, as shown in Figure \ref{fig:overlap_save_shift}. The
simulation cannot be configured to tune in the time domain at this time.

The channelizer is fully implemented in \texttt{overlap\_save\_channelizer(...)}.
The function accepts a list of center frequencies and a list of corresponding decimation factors, just like the polyphase channelizer, but it also allows the user to specify the FFT size to be used internally.

The first step performed in this function is determining the value of the
variable $P$. This corresponds to the length of the filter used, and determines
the amount of overlap in each FFT. $P$ must be greater than or equal to the
length of the longest filter and $P-1$ must be a factor of the FFT size.
A different filter must be generated for each unique decimation
factor\footnote{The filters used by the overlap-save filter bank are designed
    in the same manner as the polyphase channelizer - with a cutoff at the
nyquist frequency of the output sample rate.}, and the function assumes that
the longest filter response will be for the narrowest filter - the filter for
the largest decimation factor. So in order to determine $P$, the function first
determines the length of the filter for the largest decimation factor, then
rounds up until $P-1$ is the nearest factor of the FFT size. Every other filter
will be zero-padded to this length.

Next, the FFTs are computed (overlapped by $P-1$ samples), and they are re-used
to tune, filter and decimate each channel, as described in
Section~\ref{sec:os_filter_bank}

\subsection{Adding Cyclostationary Detection}
\label{sec:sim_os_cyclo}
As discussed in Section \ref{sec:os_advantages} a major advantage to the
Overlap-Save Filter Bank is that the Forward FFTs can be re-used to estimate
the SCD. This is implemented in the simulation function
\texttt{cyclostationary\_and\_overlap\_save(...)}. This function uses two parts of the
overlap-save channelizer separately.  First it uses \texttt{os\_fft(...)} to
determine $P$ and compute overlapped FFTs, then it uses these FFTs to perform
cyclostationary detection, and finally it re-uses the overlapped FFTs, along
with the detection information in \texttt{os\_filter(...)} to tune, filter, and
decimate each detected channel.

Note that this process uses the same \texttt{cyclostationary\_detect(...)}
function described in Section \ref{sec:sim_cyclo}, but it had to be modified to
accept either time-domain sampled data or frequency domain data. If it receives
time-domain data it performs an FFT to enter the frequency domain, but if it
receives frequency domain data it skips that step.

Decimation factors are determined to output a user configured number of samples
per symbol, just as in the Polyphase Channelizer.

Figure \ref{fig:cyclo_os_results} shows the output of this filter bank. The
high SNR of these outputs meant that every channel was demodulated with a 0\%
BER.

\begin{figure}[bh!]
    \includegraphics[width=\textwidth]{cyclo_os_results}%
\caption{Output of the combined cyclostationary detector and overlap-save filter bank. All three test signals sampled at four samples per symbol.}
\label{fig:cyclo_os_results}
\end{figure}

\chapter{Conclusion}
\label{sec:conclusion}
\markright{Brian H. Hulette \hfill Chapter \ref{sec:conclusion}. Conclusion \hfill}
Simulations of a cyclostationary detector and two separate channelizer
structures, a combination polyphase analysis/synthesis channelizer and an
overlap-save filter bank, have been presented. Both channelizer structures
were combined with the cyclostationary detector in order to evaluate their
performance when detecting and isolating QPSK signals at various symbol rates
in a wideband test signal. Both structures proved effective at accurately
detecting and isolating these signals, but only the Overlap-Save Filter Bank
was able to be combined with detection to improve computational efficiency.

Currently these simulations only uses the beginning of the test signal for
detection, but in an actual implementation the detector would be running
continuously and dynamically directing the channelizer to begin processing new
signals. It should be noted that in this described system, both channelizers would
need to have knowledge of the required signal bandwidths and output sample
rates before processing begins, so that appropriate filters can be designed, and
in the case of the polyphase analysis/synthesis channelizer, so that the size
of the analysis channelizer can be chosen.

\section{Future Work}
This work could be extended in many different ways. In this project, the
limitations of the various structures were taken as a given, however many of
these limitations can actually be avoided. For example, Harris shows that it is
in fact possible to adjust the output sample rate of the analysis channelizer
by adjusting the input commutator \cite{Harris1}. That topic was left out of this
simulation since it is difficult to devise a flexible version of it that will
work for any sample rate. Future work could focus on what output sample rates
are possible and devise an approach for a general purpose solution.

% - using single FFT to estimate cyclic spectra for non-perfect cyclic frequencies
Another limitation taken as a given is that SCD estimation with a single FFT is
impossible for cyclic frequencies that do not satisfy
Equation~\ref{eq:cyclo_freqs} - but this could potentially be overcome as well.
For cyclic frequencies, where $\alpha/2$ does not correspond to a specific FFT
bin the FFT could be interpolated before shifting. This would not be perfectly
accurate, but it would be worth evaluating how close of an approximation this
yields. Additionally, one could evaluate how complex of an interpolation could
be used while still maintaining the computational advantage of using a single FFT.

% - Evaluate Pd vs. Pfa
% - Low SNR
The only presented test case was very low noise - it would be beneficial to
evaluate how well the detector and channelizers work as a function of noise
power. A Monte Carlo test could be performed to evaluate the detector's
probability of detection, $P_d$, vs. noise power at a given threshold. Both
channelizers should also be evaluated by plotting the Bit Error Rate (BER) of
every output channel as a function of noise power. Perhaps we will find that
one of the channelizers harms the signal integrity.

% - comparison of computational complexity
Finally, it would be useful to do an in-depth analysis of the computational
complexity of both combined detector/channelizers. The theoretical
computational complexity could be determined in addition to an analysis of
simulation runtime.

% - Filter Design (TODO)

%%%%%%%%%%%%%%%%%
%
% Include an EPS figure with this command:
%   \epsffile{filename.eps}
%

%%%%%%%%%%%%%%%%
%
% Do tables like this:

% \begin{table}
% \caption{The Graduate School wants captions above the tables.}
%\begin{center}
% \begin{tabular}{ccc}
% x & 1 & 2 \\ \hline
% 1 & 1 & 2 \\
% 2 & 2 & 4 \\ \hline
% \end{tabular}
%\end{center}
% \end{table}

%%%%%%%%%%%%%%%%%%%%%%%%%%%%%%%%

% If you are using BibTeX, uncomment the following:
\nocite{*}
\bibliographystyle{IEEEtran}
\bibliography{bibliography}
%
% Otherwise, uncomment the following:
% \chapter*{Bibliography}

\appendix

% In LaTeX, each appendix is a "chapter"
\chapter{Project Source}
\label{sec:source}
\markright{Brian H. Hulette \hfill Appendix \ref{sec:source}. Project Source \hfill}
The full source is hosted on GitHub
(\url{https://github.com/TheNeuralBit/cyclo\_channelizer}). Some of the
critical files are included in this appendix.

\section{Cyclostationary Detection}
\subsection{\texttt{cyclostationary\_detect.m}}
\lstinputlisting{../cyclo/cyclo_detect.m}
\subsection{\texttt{cyclic\_spectrum.m}}
\lstinputlisting{../cyclo/cyclic_spectrum.m}

\section{Polyphase Channelizer}
\subsection{\texttt{cyclostationary\_and\_polyphase.m}}
\lstinputlisting{../cyclo_and_polyphase.m}
\subsection{\texttt{polyphase\_channelizer.m}}
\lstinputlisting{../channelize/polyphase_channelizer.m}

\section{Overlap-Save Channelizer}
\subsection{\texttt{cyclostationary\_and\_overlap\_save.m}}
\lstinputlisting{../cyclo_and_overlap_save.m}
\subsection{\texttt{overlap\_save\_channelizer.m}}
\lstinputlisting{../channelize/overlap_save_channelizer.m}
\subsection{\texttt{os\_fft.m}}
\lstinputlisting{../channelize/os_fft.m}
\subsection{\texttt{os\_filter.m}}
\lstinputlisting{../channelize/os_filter.m}

\end{document}
